%%% -*- Mode: LaTeX; -*-
%       RCS: $Id: README.tex,v 1.2 2017-12-26 16:35:03 lsipusr Exp $

%%%#+BEGIN: bx:dblock:lcnt:warning-intro :class "memo" :langs "en+fa"

%%%#+END:

%%%#+BEGINNOT: bx:dblock:lcnt:header-begin :class "memo" :langs "en+fa"
%{{{ DBLOCK-header-begin

\documentclass{article}

\usepackage{hevea} 
%HEVEA\usepackage[utf8]{inputenc}

\htmlhead{
\vspace{0.4in}
}

\htmlfoot{
\bigskip
}


\usepackage{fontspec}
\setmainfont[Mapping=tex-text]{Linux Libertine O}

\usepackage{morefloats}

\usepackage{rcs}
\usepackage{makeidx}
\usepackage{supertabular}
\usepackage{lscape}
\usepackage{array} 
\usepackage{framed}
\usepackage{listings}

\usepackage{color}

\usepackage{hyperref}
\usepackage{url}

\usepackage{fancyhdr}

\usepackage{caption}

\usepackage{fontspec}
\usepackage{xltxtra}
\usepackage{xunicode}

%}}} DBLOCK-header-begin
%%%#+END:

%%%#+BEGIN: bx:dblock:lcnt:style-params :class "memo" :langs "en+fa"
\begin{comment}
*  [[elisp:(org-cycle)][| ]]  *DBLK: style-params*                                       :: [[elisp:(beginning-of-buffer)][Top]] [[elisp:(delete-other-windows)][(1)]]  [[elisp:(org-cycle)][| ]]
\end{comment}
% ===== STYLE PARAMETERS =====

\definecolor{darkred}{rgb}{0.5,0,0}
\definecolor{darkgreen}{rgb}{0,0.5,0}
\definecolor{darkblue}{rgb}{0,0,0.5}

\hypersetup{
    bookmarks=true,         % show bookmarks bar?
    unicode=false,          % non-Latin characters in Acrobat’s bookmarks
    pdftoolbar=true,        % show Acrobat’s toolbar?
    pdfmenubar=true,        % show Acrobat’s menu?
    pdffitwindow=false,     % window fit to page when opened
    pdfstartview={FitH},    % fits the width of the page to the window
    pdftitle={My title},    % title
    pdfauthor={Author},     % author
    pdfsubject={Subject},   % subject of the document
    pdfcreator={Creator},   % creator of the document
    pdfproducer={Producer}, % producer of the document
    pdfkeywords={keyword1} {key2} {key3}, % list of keywords
    pdfnewwindow=true,      % links in new window
    colorlinks=true ,       % false: boxed links; true: colored links
    linkcolor=darkblue,     % color of internal links
    citecolor=red,          % color of links to bibliography
    filecolor=darkgreen,    % color of file links
    urlcolor=darkred        % color of external links
}


\setlength{\textwidth}{6.0in}
\addtolength{\oddsidemargin}{-0.75in}
\addtolength{\evensidemargin}{-0.75in}

\topmargin      0.00 in
\textheight     8.50 in

\setlength{\textwidth}{16.5cm}
\setlength{\topmargin}{-0.3in}
\setlength{\textheight}{8.5in}
\setlength{\oddsidemargin}{0.0cm}
\setlength{\evensidemargin}{0.0cm}


\pagestyle{fancy}
\fancyhead{} % clear all header fields  
%% \fancyhead[C]{{\small  {\tt Work In Progress}}}
\renewcommand{\headrulewidth}{0pt} % no line in header area
\fancyfoot{} % clear all footer fields
%%\fancyfoot[LE,RO]{\thepage}           % page number in "outer" position of footer line
%% \fancyfoot[RE,LO]{{\tt --EARLY DRAFT DOCUMENT--\hspace{20 mm} --Reflects Work In Progress-- }}
\fancyfoot[RE,LO]{}


\parindent 0 true pc

\addtolength{\parskip}{5pt}


%%%#+END:

%%%#+BEGIN: bx:dblock:lcnt:header-end :class "memo" :langs "en+fa"
%{{{ DBLOCK-header-end

\begin{document}
%}}} DBLOCK-header-end
%%%#+END:

%%%#+BEGIN: bx:dblock:lcnt:front-begin :class "memo" :langs "en+fa"
\begin{comment}
*  [[elisp:(org-cycle)][| ]]  *DBLK: front-begin*                                       :: [[elisp:(beginning-of-buffer)][Top]] [[elisp:(delete-other-windows)][(1)]]  [[elisp:(org-cycle)][| ]]
\end{comment}

%%%#+END:

%%%#+BEGIN: bx:dblock:lcnt:copyright :class "memo" :langs "en+fa"
\begin{comment}
*  [[elisp:(org-cycle)][| ]]  *DBLK: copyright*                                       :: [[elisp:(beginning-of-buffer)][Top]] [[elisp:(delete-other-windows)][(1)]]  [[elisp:(org-cycle)][| ]]
\end{comment}

%%%#+END:

%%%#+BEGIN: bx:dblock:lcnt:front-end :class "memo" :langs "en+fa"
\begin{comment}
*  [[elisp:(org-cycle)][| ]]  *DBLK: front-end*                                       :: [[elisp:(beginning-of-buffer)][Top]] [[elisp:(delete-other-windows)][(1)]]  [[elisp:(org-cycle)][| ]]
\end{comment}

%%%#+END:

%%%#+BEGINNOT: bx:dblock:lcnt:main-begin :class "memo" :langs "en+fa"
\begin{comment}
*  [[elisp:(org-cycle)][| ]]  *DBLK: main-begin*                                       :: [[elisp:(beginning-of-buffer)][Top]] [[elisp:(delete-other-windows)][(1)]]  [[elisp:(org-cycle)][| ]]
\end{comment}

\title{unisos.wsf}

%%%#+END:

\thispagestyle{empty}


\bigskip

\section{Overview}

unisos.wsf: Somewhat General purpose Web Scraping Framework (WSF)


\section{Support}

For support, criticism, comments and questions; please contact the 
author/maintainer \\
\href{http://mohsen.1.banan.byname.net}{Mohsen Banan} at: \url{http://mohsen.1.banan.byname.net/contact}


\section{Documentation}

The is no silver bullet for universally providing web scraping capabilities.

Each web site is different and various technologies, encodings and paradigms are used.

What can be done is to create a framework that addresses some of the common aspects of this problem domain.

This is what Web Scraping Framework (WSF) tries to do.

Common aspects of Web scraping include:

\begin{description}
  \item[Configuration:]
        WSF uses python function invocation as the configuration syntax.
\begin{verbatim}
        See ./unisos/wsf/wsf_config.py for details.
\end{verbatim}

  \item[Simultaneous Parallel Dispatch Of Multiple Urls:]
        WSF provides a rudimentary mechanism for parallel dispatch.
\begin{verbatim}
        See ./unisos/wsf/wsf_parallelProc.py for details.
\end{verbatim}
        Use of external systems such as celery for workers dispatch is a better solution for
        larger systems.

  \item[Retrieving Web Content As html:]
        WSF uses requests to obtain html.
\begin{verbatim}
        See ./unisos/wsf/wsf_inputs.py for details.
\end{verbatim}

  \item[Digesting html:]
        WSF uses Beautiful Soup 4 to digest html.
\begin{verbatim}
        See ./unisos/wsf/wsf_digestHtml.py for details.
\end{verbatim}

  \item[Basic Scraping Facilities:]
        WSF provides some generic facilities for basic scripting as an abstract class.
\begin{verbatim}
        See ./unisos/wsf/wsf_scraperBasic.py for details.
\end{verbatim}

        The ScraperBasic is an abstract and incomplete class which must be subclassed
        to become concrete.

        Features provided by this principal class are:
        \begin{itemize}
          \item Capturing of Config parameters.
          \item Facilities for simple state transition.
          \item Facilities for maintaining results.

        \end{itemize}

  \item[Multipage Scraping Facilities:]
        WSF provides some facilities for web information that has been paginated.
\begin{verbatim}
        See ./unisos/wsf/wsf_scraperMultipage.py for details.
\end{verbatim}

        The ScraperMultipage(wsf\_scraperBasic.ScraperBasic) is an abstract and incomplete class which must be subclassed
        to become concrete.

        ScraperMultipage itself is a subclass of wsf\_scraperBasic.ScraperBasic.


  \item[Capturing Results And Writing Results:]
        WSF provides a rudimentary mechanism for capturing intermediate results and their output.
\begin{verbatim}
        See ./unisos/wsf/wsf_results.py for details.
\end{verbatim}

  \item[Command Line Mapping:]
        WSF can be used in combination with unisos.icm (Interactive Command Modules)

        ICM can be thought of as a supper set of click which supports plugins as
        ``load'' parameters.

        Both config files and concrete scraper classes can be passed to ICMs as ``load''
        parameters.

\end{description}

\section{Installation}

\subsection{From PyPi}

\begin{verbatim}
pip install unisos.wsf
\end{verbatim}

\subsection{From File System}

Go to the wsf/py3 directory.

\begin{verbatim}
Run: ./setup.py sdist
\end{verbatim}

\begin{verbatim}
Run: pip install --no-cache-dir ./dist/unisos.wsf-0.1.tar.gz
\end{verbatim}


\section{Usage}

\begin{verbatim}
import unisos.wsf
\end{verbatim}

Use of unisos.wsf involves creating concrete subclasses of the
set of abstract classes that wsf provides.


%%%#+BEGIN: bx:dblock:lcnt:main-end :class "memo" :langs "en+fa"
\begin{comment}
*  [[elisp:(org-cycle)][| ]]  *DBLK: main-end*                                       :: [[elisp:(beginning-of-buffer)][Top]] [[elisp:(delete-other-windows)][(1)]]  [[elisp:(org-cycle)][| ]]
\end{comment}

\end{document}

%%%#+END:

%%%#+BEGIN: bx:dblock:lcnt:latex:end-of-file :class "memo" :langs "en+fa"
%local variables:
%major-mode: latex-mode
%folded-file: nil
%fill-column: 65
%TeX-master: ""
%End:
%%%#+END:
